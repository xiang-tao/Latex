\documentclass{ctexart}

\usepackage{lipsum}%ダミーの文章を入れる
\usepackage{capt-of}
\usepackage[margin=2cm]{geometry}
\usepackage{tcolorbox}
\usepackage{varwidth}
\usepackage{amsmath}
\tcbuselibrary{breakable}
\tcbuselibrary{skins}
\tcbuselibrary{xparse}%xparse . sty定义的NewDocumentCommand
\definecolor{frameinnercolor}{RGB}{49,44,44}


\newcounter{theorem}
\numberwithin{theorem}{section}% numberwithin是amsmath定义
\NewTColorBox{theobox}{o m +o}{% o省略可能的参数、m必备的参数、+是多行里,也可以表示。
%# 1 =标题(省略可), # 2 =定理环境名,# 3 = tcolorbox追加的选项
enhanced,frame empty,interior empty,
coltitle=white,fonttitle=\bfseries,colbacktitle=frameinnercolor,
extras broken={frame empty,interior empty},
borderline={0.5mm}{0mm}{frameinnercolor},
sharp corners=downhill,
breakable=true,
top=4mm,
before skip=3.5mm,
attach boxed title to top left={yshift=-3mm,xshift=5mm},
boxed title style={boxrule=0pt,sharp corners=all},varwidth boxed title,
IfNoValueTF={#1}{title=#2~\thetheorem.}{title=#2~\thetheorem:~#1},
IfNoValueTF={#3}{}{#3}%第三个参数省略- NoValue -返回
}

\NewDocumentEnvironment{theorem}{o m +o}{%NewDocumentEnvironment由xparse.sty定义
\refstepcounter{theorem}\begin{theobox}[#1]{#2}[#3]}{\end{theobox}}

\NewDocumentEnvironment{theo}{o +d""}{%d""表示作为省略的一部分。
\begin{theorem}[#1]{定理}[#2]
}{\end{theorem}}

\NewDocumentEnvironment{law}{o +d""}{%
\begin{theorem}[#1]{法则}[#2]
}{\end{theorem}}


\begin{document}

\section{测试定理}

\begin{theo}[Title]
 \lipsum[1]
\end{theo}

\begin{law}[库仑定律]"sidebyside, 
sidebyside align=top seam,
sidebyside gap=5mm,
righthand width=4.4cm,
lower separated=false,% 
halign lower=flush right,
"%
查尔斯·库仑1785年,
收取称重小球之间起作用的扭力测量,
带电粒子之间作用的力正比于各电荷的产物,
成反比的距离平方(按平方反比定律)日前宣布,它(以下公式)。
\begin{equation*}
 F=k\frac{q_1 q_2}{r^2} \quad
\left(
\begin{array}{ll}
  q_1, q_2\text{充带电粒子的电荷量}\\[3pt]
 r\text{粒子距离}
\end{array}
\right)
\end{equation*}

用该公式计算时,不要把电荷的正负符号代入公式中,计算过程可用绝对值计算,可根据同名电荷相斥,异名电荷相吸来判断力的方向。

在库仑定律的常见表述中,通常会有真空和静止,是因为库仑定律的实验基础——扭秤实验,为了排除其他因素的影响,是在亚真空中做的。另外,一般讲静电现象时,常由真空中的情况开始,所以库仑定律中有“真空”的说法。实际上,库仑定律不仅适用于真空中,还适用于均匀介质中,也适用于静止的点电荷之间。

\tcblower
 \includegraphics[width=4.6cm]{lyf.jpg}
\captionof{figure}{库仑扭秤}
\end{law}

库仑定律只适用于点电荷之间。带电体之间的距离比它们自身的大小大得多,以至形状、大小及电荷的分布状况对相互作用力的影响可以忽略,在研究它们的相互作用时,人们把它们抽象成一种理想的物理模型——点电荷,库仑定律只适用于点电荷之间的受力。

\end{document} 