\documentclass{article}
\usepackage{ctex}
\usepackage{pythonhighlight}
\usepackage{amsmath,amssymb,amsfonts}
\usepackage{geometry}
\geometry{left=2cm,right=2cm,top=2.5cm,bottom=2.5cm}
\usepackage[utf8]{inputenc}

\title{数值计算程序设计作业}
\author{向~涛}
\date{July 2021}

\begin{document}

\maketitle

\section{散度定理}

给定区域$\Omega \in \mathbb{R}^d$,记$\partial \Omega$为$\Omega$的边界,$\pmb{n}$为$\partial \Omega$为$\Omega$的单位外法向量,$u$和$v$是标量函数,$\pmb{u}$和$\pmb{v}$是向量函数,利用散度定理证明以下公式成立:

\begin{equation}
    \int_{\Omega} \nabla u \cdot \pmb{v} d \pmb{x} + \int_{\Omega} u \nabla \cdot\pmb{v} d \pmb{x} 
    =\int_{\partial\Omega} u \pmb{n} \cdot \pmb{v} d \pmb{x}
\end{equation}

证明:设$\pmb{F}=u\pmb{v}$,则有

\begin{align*}
    \nabla \cdot u \pmb{v} = \sum^{d}_{i=1} \dfrac{\partial(uv_i)}{\partial x_i} &= \sum^{d}_{i=1} 
    \dfrac{\partial u}{\partial x_i} v_i + u \sum^{d}_{i=1}
    \dfrac{\partial v_i}{\partial x_i}  \\
    &= \nabla u \cdot  \pmb{v} + u\nabla \cdot \pmb{v} 
\end{align*}

此时便有:

\begin{equation*}
    \int_{\partial\Omega}u\pmb{n}\cdot\pmb{v}d\pmb{x} = \int_{\partial\Omega}\pmb{n}\cdot u\pmb{v}d\pmb{x} = \int_{\Omega}\nabla \cdot u \pmb{v}d\pmb{x}=\int_{\Omega} \nabla u \cdot \pmb{v} d \pmb{x} + \int_{\Omega} u \nabla \cdot\pmb{v} d \pmb{x}
\end{equation*}

\begin{equation}
    \int_{\Omega}\nabla\times \pmb{u}\cdot \pmb{v} d\pmb{x}-\int_{\Omega} \pmb{u}\cdot\nabla\times \pmb{v} d\pmb{x} =\int_{\partial\Omega}\pmb{n}\times \pmb{u}\cdot \pmb{v} d\pmb{x}
\end{equation}

证明:根据$\nabla$$\cdot$$\left(\pmb{u}\times \pmb{v}\right)=\nabla\times \pmb{u}\cdot \pmb{v}-\pmb{u}\cdot\nabla \times \pmb{v}$则有\[\int_{\Omega}\nabla\times \pmb{u}\cdot \pmb{v}d\pmb{x}-\int_{\Omega}u\cdot\nabla\times \pmb{v} d\pmb{x}=\int_{\Omega}\nabla\cdot\left(\pmb{u}\times	 \pmb{v}\right)d\pmb{x}\]
设$F=\pmb{u}\times \pmb{v}$,根据散度定理有
\[\int_{\Omega}\nabla\cdot \left(\pmb{u}\times \pmb{v}\right)d\pmb{x}=\int_{\partial\Omega}\pmb{u}\times \pmb{v}\cdot \pmb{n}d\pmb{x}\]
于是便有\[\int_{\Omega}	\nabla\times \pmb{u}\cdot \pmb{x} d\pmb{x}-\int_{\Omega}\pmb{u}\cdot\nabla\times \pmb{v} d\pmb{x}=\int_{\Omega}\left(\pmb{u}\times \pmb{v}\cdot \pmb{n}\right)d\pmb{x}\]

\begin{equation}
    \int_{\Omega}\nabla\cdot \pmb{u} vd\pmb{x}+\int_{\Omega}u\cdot\nabla vd\pmb{x}=\int_{\partial \Omega}\pmb{n}\cdot \pmb{u}v d\pmb{x}
\end{equation}

证明:根据散度定理\[\int_{\Omega}\nabla\cdot \pmb{u} vd\pmb{x}+\int_{\Omega}\pmb{u}\cdot\nabla vd\pmb{x}=\int_{\Omega}\nabla\cdot\left(\pmb{u}v\right)=\int_{\partial}\nabla\left(\pmb{u}v\right) d\pmb{x}\]
于是有\[\int_{\Omega}\nabla\left(\pmb{u}v\right)d\pmb{x}=\int_{\partial \Omega}\pmb{n}\cdot \pmb{u}vd\pmb{x}=\int_{\Omega}\nabla\cdot \pmb{u} vd\pmb{x}+\int_{\Omega}\pmb{u}\cdot\nabla vd\pmb{x}\]

\section{edge2cell}

仿照三角形edge2cell程序编写四边形的edge2cell程序;四边形的edge2cell程序如下:

\begin{python}
import numpy as np
from fealpy.mesh import QuadrangleMesh
import matplotlib.pyplot as plt


def arrayprint(name, a):
    """
    Note
    ----
    打印一个名字为 name 的数组 a,每一行之前加一个行号
    """
    print("\n", name + ":")
    for (i, row) in enumerate(a):
        print(i, ": ", row)


node = np.array([(0, 0), (1, 0), (1, 1), (0, 1)], dtype=np.float)
cell = np.array([(0, 1, 2, 3)], dtype=np.int)
mesh = QuadrangleMesh(node, cell)
mesh.uniform_refine(1)

node = mesh.entity('node')
cell = mesh.entity('cell')

# 从 cell 出发,构造 edge、edge2cel、cell2edge
NC = mesh.number_of_cells()
NEC = 4  # 每个单元有4条边
localEdge = np.array([
    [0, 1],  # 局部 0 号边
    [1, 2],  # 局部 1 号边
    [2, 3],  # 局部 2 号边
    [3, 0],  # 局部 3 号边
], dtype=np.int_)  # (4, 2)
# (NC, 4)---> (NC, 4, 2) --> (4*NC, 2)
totalEdge = cell[:, localEdge].reshape(-1, 2)
stotalEdge = np.sort(totalEdge, axis=-1)

"""
这里对python的np.unique函数作说明:
1.a = np.unique(A)
对于一维数组或者列表,unique函数去除其中重复的元素,并按元素由大到小返回一个新的无元素重复的元组或者列表
2.c,s=np.unique(b,return_index=True)
return_index=True表示返回新列表元素在旧列表中的位置,并以列表形式储存在s中。
3.a, s,p = np.unique(A, return_index=True, return_inverse=True)
return_inverse=True 表示返回旧列表元素在新列表中的位置,并以列表形式储存在p中
"""
sedge, i0, j = np.unique(stotalEdge,
                         axis=0,
                         return_index=True,
                         return_inverse=True)

i1 = np.zeros_like(i0)
i1[j] = range(NEC * NC)  # 取重复元素的后者

edge = totalEdge[i0]  # 去除重复元素后将顺序复原得到边的约定数据
cell2edge = j.reshape(-1, NEC)  # (NC, 4)

NE = len(edge)
edge2cell = np.zeros((NE, 4), dtype=np.int_)
edge2cell[:, 0] = i0 // NEC
edge2cell[:, 1] = i1 // NEC
edge2cell[:, 2] = i0 % NEC
edge2cell[:, 3] = i1 % NEC

# 输出构造出的edge cell2edge edge2cell

arrayprint("edge", edge)
arrayprint("edge2cell", edge2cell)
arrayprint("cell2edge", cell2edge)

fig = plt.figure()
axes = fig.gca()
mesh.add_plot(axes)
mesh.find_node(axes, showindex=True)
mesh.find_cell(axes, showindex=True)
mesh.find_edge(axes, showindex=True)
plt.show()
\end{python}

\section{数值积分公式}

利用fealpy中数值积分计算函数在区域上的积分,并求和得到在整个区域上的积分。

本次代码实现的数值代数积分函数是$f(x,y)=2(x+y)$,在边长为1的矩形区域上的积分为$\int_0^1\int_0^1 2(x+y)dxdy=2$,那么当分为两个三角形区域时,由于函数的对称性,则在每个区域上的积分为1,经程序验证是一致的,其实现的代码如下:

\begin{python}
"""
Notes
-----
任给一个或者一组重心坐标点,计算出网格中每个单元上对应的笛卡尔坐标
并计算函数在区域上的数值积分

(1/3, 1/3, 1/3)   --->  (NC, 2)

"""
import numpy as np
from fealpy.mesh import MeshFactory as MF
import matplotlib.pyplot as plt

box2d = [0, 1, 0, 1]
mesh = MF.boxmesh2d(box2d, nx=1, ny=1, meshtype='tri')
NC = mesh.number_of_cells()  # NC=2
node = mesh.entity('node')
cell = mesh.entity('cell')  # shape = (NC,3)
v0 = node[cell[:, 2]] - node[cell[:, 1]]  # (NC,2)
v1 = node[cell[:, 0]] - node[cell[:, 2]]  # (NC,2)
v2 = node[cell[:, 1]] - node[cell[:, 0]]  # (NC,2)
S = 1 / 2 * np.cross(v1, v2)  # nv = np.cross(v2, -v1), shape=(NC,)
bc = np.array([1 / 3, 1 / 3, 1 / 3], dtype=mesh.ftype)  # 重心坐标
qf = mesh.integrator(1)  # 一致加密
bcs, ws = qf.quadpts, qf.weights  # 重心坐标积分点和权重
ps1 = mesh.bc_to_point(bcs)
ps = mesh.bc_to_point(bc)

"""

Notes:
-----
用高斯积分计算如下的一个简单函数,f(x,y)=2(x+y)
显然此时该函数在该区域上的积分值为2,一共两个单元,
故其中每个单元的积分值为1

"""


def f(x, y):
    return 2*(x+y)


val = np.zeros((1, NC))
x = np.zeros((len(ps1)))
y = np.zeros((len(ps1)))
valf = np.zeros((len(ps1)))

for i in range(NC):
    x = ps1[:, i, 0]
    y = ps1[:, i, 1]
    valf = f(x, y)
    val[0, i] = S[i] * ws.reshape(1, -1) @ valf.reshape(-1, 1)

print('各个单元上的积分值:', val)
print('整个区域上的积分值:', np.sum(val))
\end{python}

\section{重心坐标}

仿照三角形重心坐标的程序编写四面体重心坐标生成的程序。

理论分析:四面体有四个顶点,对应的坐标分别设为$\pmb{x_0},\pmb{x_1},\pmb{x_2},\pmb{x_3}$,对应的重心坐标函数分别设为$\lambda_0,\lambda_1,\lambda_2,\lambda_3$,设顶点$\pmb{x_0}$对应的面的面积为$S$,该点到对应面的高为$h$,易知此时重心坐标函数$\lambda_0$的梯度方向就是高的方向,设$\pmb{n}$为与高平行的单位方向向量,则可求得方向导数为:

\begin{equation*}
    \nabla \lambda_0(\pmb{x})\cdot\pmb{n}=\dfrac{1}{h}
\end{equation*}

上式中的$\dfrac{1}{h}$是由于当$\pmb{x}$在点$\pmb{x_0}$对应的面上时候,$\lambda_0(\pmb{x_0})=0$,当$\pmb{x}$与点$\pmb{x_0}$重合时候有$\lambda(\pmb{x})=1$,他们之间的距离为$h$,故有方向导数的值为$\dfrac{1}{h}$,同事由于$\pmb{n}$是单位向量,那么可得到该梯度的大小也为$\dfrac{1}{h}$,有了梯度的大小和方向,就可以确定梯度形式

\begin{align*}
    \nabla\lambda_0 = \dfrac{1}{h}\pmb{n} 
                    =\dfrac{S}{V}\pmb{n}  
                    =\dfrac{1}{2V}(\pmb{x_3}-\pmb{x_1})\times(\pmb{x_2}-\pmb{x_1})
\end{align*}

同理可得另外三个重心坐标函数的梯度形式,实现的代码如下:

\begin{python}
"""
Notes:
-----
仿照三角形求解四面体的重心坐标函数的梯度

"""
import numpy as np
from fealpy.mesh import MeshFactory as MF

box3d = [0, 1, 0, 1, 0, 1]
mesh = MF.boxmesh3d(box3d, nx=1, ny=1, nz=1, meshtype='tet')
NC = mesh.number_of_cells()
node = mesh.entity('node')
cell = mesh.entity('cell')
v0 = node[cell[:, 0], :] - node[cell[:, 1], :]
v1 = node[cell[:, 0], :] - node[cell[:, 2], :]
v2 = node[cell[:, 0], :] - node[cell[:, 3], :]
v3 = node[cell[:, 2], :] - node[cell[:, 1], :]
v4 = node[cell[:, 3], :] - node[cell[:, 2], :]
v5 = node[cell[:, 1], :] - node[cell[:, 3], :]
mes = np.einsum('ij,ij->i', v1, np.cross(v3, v4))  # 爱因斯坦求和
Dlambda = np.zeros((NC, 4, 3), dtype=np.float64)
Dlambda[:, 0, :] = np.cross(v3, v4) / mes.reshape(-1, 1)
Dlambda[:, 1, :] = np.cross(-v4, v2) / mes.reshape(-1, 1)
Dlambda[:, 2, :] = np.cross(v2, -v0) / mes.reshape(-1, 1)
Dlambda[:, 3, :] = np.cross(-v3, v0) / mes.reshape(-1, 1)
print("Dlambda:", Dlambda)
\end{python}

\end{document}
